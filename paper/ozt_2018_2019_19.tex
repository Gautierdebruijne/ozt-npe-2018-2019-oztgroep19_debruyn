\documentclass{article}

%opening
\title{Literatuurstudie: OZT 2018-2019}
\date{07/03/2019}
\author{Santi Meremans \\ Bono Opalfvens \\ Fé Vanmanshoven \\ Gautier de Bruijne \\\\ Hogeschool Gent \\ Toegepaste Informatica \\ Groep 19, Dhr. De Bruyne}

\begin{document}

\maketitle
\pagenumbering{gobble}
\newpage

\renewcommand*\contentsname{Inhoudstafel}

\tableofcontents
\newpage

\pagenumbering{arabic}
\section{Inleiding}
In het kader van onderzoekstechnieken hebben we verschillende artikels i.v.m. "Long term retention" onderzocht en vergeleken. Uit deze artikels hebben we de hoofdzaken geselecteert en samengevoegd in onze studie, zodat we de voor- en nadelen van de verschillende onderzoeken kunnen bekijken en tot een efficiënte manier van werken kunnen komen.

\subsection{Opdracht}

\section{Conceptueel model}
\subsection{Onderzoeksvragen}
\subsection{Hypothesen}
\subsection{Onderzoeksmethoden}

\section{Literatuurstudie}
\subsection{Efficiënt studeren}
\subsubsection{Iteratief}
\subsubsection{Vocaal}
\subsubsection{Kennisproeven}

\end{document}


